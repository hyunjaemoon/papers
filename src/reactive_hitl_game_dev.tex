\documentclass{article}
\usepackage{geometry}
\geometry{a4paper, margin=1in}
\usepackage{amsmath}
\usepackage{amssymb}
\usepackage{graphicx}
\usepackage{hyperref}

\title{Designing Free-Form Stories with LLMs in Video Game Development: Reactive Human-in-the-Loop and Agentic Workflows}
\author{Hyun Jae Moon \\ \small Software Engineer \\ \small \href{mailto:calhyunjaemoon@gmail.com}{calhyunjaemoon@gmail.com}}
\date{March 3, 2025}

\begin{document}

\maketitle

\begin{abstract}
This paper explores the design of free-form stories in video games using Large Language Models (LLMs), emphasizing the critical role of Reactive Human-in-the-Loop (RHIT) and agentic workflows.  We argue that while LLMs offer unprecedented potential for generating dynamic and branching narratives, their effective integration into game development necessitates a RHIT approach to ensure coherence, player agency, and artistic direction.  We propose an agentic workflow framework that leverages RHIT for free-form story design, discuss relevant benchmarks for evaluating LLM-generated story quality, and explore the implications for future game narrative design.
\end{abstract}

\section{Introduction}

Video game narratives are evolving beyond linear scripts towards dynamic and branching storylines, offering players greater agency and immersive experiences. Large Language Models (LLMs) present a transformative opportunity in this evolution, capable of generating vast amounts of text and adapting narratives in real-time based on player actions. However, the unconstrained generative power of LLMs poses challenges in maintaining narrative coherence, artistic vision, and meaningful player impact. This paper posits that a Reactive Human-in-the-Loop (RHIT) approach, integrated with agentic workflows, is essential for harnessing LLMs to design compelling free-form stories in video games.  RHIT allows for human guidance and intervention at critical junctures, ensuring narrative quality and alignment with design goals, while agentic workflows streamline the collaborative process between LLMs and game developers.

\section{Literature Review: LLMs in Game Development and Interactive Narrative}

The application of LLMs in game development is a burgeoning field, with research exploring their use in dialogue generation, quest design, and world-building.  Interactive narrative, a long-standing area of game research, seeks to create player-driven stories with meaningful choices and consequences.  Existing approaches to interactive narrative often rely on pre-authored branching trees or procedural generation techniques with limited narrative depth. LLMs offer the potential to move beyond these constraints, generating novel narrative content dynamically.  However, the challenge lies in directing LLMs to produce stories that are not only reactive but also engaging, coherent, and thematically resonant.  Agentic AI, where multiple AI agents collaborate to achieve complex tasks, provides a framework for managing the complexity of LLM-driven narrative generation.  Furthermore, Human-in-the-Loop (HITL) systems are recognized as crucial for guiding and refining AI outputs, particularly in creative domains.  This paper focuses on a specific HITL paradigm, Reactive Human-in-the-Loop, where human intervention is triggered by system-defined events or performance metrics, offering a balance between automation and human oversight.

\section{Design and Implementation: Agentic Workflow with Reactive Human-in-the-Loop}

We propose an agentic workflow for free-form story design that integrates RHIT at key stages of the narrative generation process.  This workflow comprises several interacting agents:

\begin{itemize}
    \item \textbf{Narrative Generator Agent (NGA):}  This agent leverages a pre-trained LLM to generate narrative text, responding to player actions and game events.  It is responsible for creating scenes, dialogues, and descriptive text that advance the story.
    \item \textbf{Coherence and Consistency Agent (CCA):} This agent monitors the NGA's output for narrative coherence, character consistency, and adherence to the established game world lore.  It flags potential inconsistencies or plot holes.
    \item \textbf{Reactive Human-in-the-Loop (RHIT) Interface:} This interface is triggered by the CCA when narrative issues are detected or by predefined game events (e.g., reaching a critical plot point).  It presents human narrative designers with the LLM-generated text and flags, allowing for real-time editing, rewriting, or redirection of the narrative.
    \item \textbf{Player Input Agent (PIA):} This agent processes player actions and choices within the game, translating them into narrative context for the NGA.  It ensures player agency is reflected in the evolving story.
    \item \textbf{Benchmark and Evaluation Agent (BEA):}  This agent continuously evaluates the generated story against predefined benchmarks (discussed below), providing feedback to the NGA and CCA and informing RHIT interventions.
\end{itemize}

The workflow operates iteratively: the PIA captures player input, the NGA generates narrative content, the CCA assesses coherence, and RHIT intervenes when necessary. The BEA provides continuous feedback, allowing for iterative refinement of both the LLM and the agentic system.  The RHIT interface is designed to be *reactive*, meaning it only engages human designers when automated agents identify potential issues or critical decision points, maximizing efficiency while ensuring quality control.

\section{Benchmarks for LLM-Generated Story Writing in Games}

Evaluating the quality of LLM-generated stories in games requires benchmarks beyond traditional text generation metrics like perplexity or BLEU score. We propose a multi-faceted benchmark framework encompassing:

\begin{itemize}
    \item \textbf{Narrative Coherence and Consistency:}  Measures the logical flow of the story, the consistency of characters and world lore, and the absence of plot holes.  This can be assessed through automated metrics (e.g., entity linking, relation extraction) and human evaluation.
    \item \textbf{Player Engagement and Immersion:}  Assesses how effectively the story captures and maintains player interest. Metrics include player feedback surveys, playtesting data (e.g., playtime, choice selections), and physiological measures (e.g., engagement metrics via biometrics in controlled studies).
    \item \textbf{Meaningful Player Agency:}  Evaluates the extent to which player choices genuinely impact the narrative and game world.  This can be measured by tracking the branching paths taken by players and assessing the perceived impact of their decisions on the story outcome.
    \item \textbf{Emotional Resonance and Thematic Depth:}  Assesses the story's ability to evoke emotions and explore meaningful themes.  This is primarily evaluated through qualitative human assessment, focusing on the narrative's artistic merit and impact.
    \item \textbf{Authorial Intent Alignment:} Measures how well the generated story aligns with the game developers' intended narrative goals and artistic vision, assessed through expert human evaluation, particularly focusing on the effectiveness of RHIT interventions.
\end{itemize}

These benchmarks provide a comprehensive framework for evaluating the success of LLM-driven free-form story design, moving beyond purely quantitative metrics to encompass qualitative aspects of narrative quality and player experience.

\section{Discussion and Future Research}

The proposed RHIT-driven agentic workflow offers a promising approach to designing free-form stories with LLMs in video games.  The reactive nature of the human intervention ensures efficient use of developer time while maintaining crucial creative control.  Future research should focus on:

\begin{itemize}
    \item \textbf{Empirical Validation:}  Implementing and testing the proposed framework in a prototype game, evaluating its performance against the proposed benchmarks through playtesting and user studies.
    \item \textbf{RHIT Interface Design:}  Investigating optimal interface designs for RHIT, focusing on usability and efficiency for narrative designers to effectively guide LLM outputs.
    \item \textbf{Adaptive Benchmarking:}  Developing dynamic benchmarks that adapt to different game genres and narrative styles, allowing for more nuanced evaluation of LLM-generated stories.
    \item \textbf{Ethical Considerations:}  Exploring the ethical implications of LLM-driven narrative generation, particularly concerning authorship, player manipulation, and the potential for biased or harmful content.
\end{itemize}

Further exploration into these areas will be crucial for realizing the full potential of LLMs in creating truly dynamic and engaging game narratives.

\section{Conclusion}

Designing free-form stories with LLMs in video games presents both immense opportunities and significant challenges.  This paper argues that Reactive Human-in-the-Loop, integrated within an agentic workflow, is a key ingredient for successfully harnessing LLMs to create compelling and player-driven narratives.  By combining the generative power of LLMs with human artistic direction and quality control, and by employing robust benchmarks for evaluation, we can move towards a future where video games offer truly dynamic and personalized storytelling experiences.

\end{document}